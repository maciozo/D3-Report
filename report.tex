%%%%%%%%%%%%%%%%%%%%%%%%%%%%%%%%%%
% EL/EEE D1 Report Template
% University of Southampton
%
% author : Rhys Thomas (rt8g15)
%
% edited : 2016-11-14
%%%%%%%%%%%%%%%%%%%%%%%%%%%%%%%%%%

\documentclass[a4paper,11pt]{article}

%%%%%%%%%%%%%%%%%%%%%%%%%%%%%%%%%%
% PACKAGES
%%%%%%%%%%%%%%%%%%%%%%%%%%%%%%%%%%
\usepackage[margin=1in]{geometry}
\renewcommand{\baselinestretch}{1.2} % line spacing
\usepackage{color}
\usepackage{siunitx}
\usepackage{graphicx}
\usepackage{epstopdf}
\usepackage{float}
\usepackage{hyperref}
\usepackage{mathtools}
\usepackage[titletoc,toc,title]{appendix}
\usepackage{subfiles}
\usepackage{pgfplots}
\usepackage[european]{circuitikz}
\usepackage{textgreek}
\usepackage{amssymb}

\pgfplotsset{compat=1.13}
\pgfplotsset{unit code/.code={\si{#1}}}
\usepgfplotslibrary{units}

\graphicspath{ {./images/} }

\usepackage{newunicodechar}
\newunicodechar{p}{\ifmmode\pi\else\textpi\fi}

%%%%%%%%%%%%%%%%%%%%%%%%%%%%%%%%%%
% DOCUMENT BEGIN
%%%%%%%%%%%%%%%%%%%%%%%%%%%%%%%%%%
\begin{document}
  
\begin{center}
{\Large{\textbf{ELEC2205 D3 -- Two-Stage Amplifier Design}}} \\ [\baselineskip]
\subfile{info.tex}
\end{center}

\begin{abstract}
\end{abstract}

\section{Theoretical Design}
\subsection{Voltage Gain Derivation}
\subsubsection{Stage 1}

\begin{figure}[h]
\centering
    \subfile{stage1Circuit.tex}
    \caption{Circuit diagram of the first stage of the amplifier.}
    \label{fig:stage1}
\end{figure}

In order to derive the voltage gain of this circuit, we need to analyse it using a small signal model. The hybrid-\textpi\ model (figure~\ref{fig:stage1hpi}) will be used.

\begin{figure}[h]
\centering
    \subfile{stage1hpi.tex}
    \caption{Hybrid-\textpi\ model of the first stage of the amplifier.}
    \label{fig:stage1hpi}
\end{figure}

Using Kirchoff's current law on the base:

\begin{subequations}
\begin{align}
\frac{v_b - v_s}{R_s} + \frac{v_b}{R_2} + \frac{v_b}{R_1} + \frac{v_b - v_e}{r_\pi} &= 0\\
\frac{v_b - v_s}{R_s} + v_b \left(\frac{1}{R_1} + \frac{1}{R_2} \right) + \frac{v_b - v_e}{r_\pi} &= 0 \label{eq:kclBase}
\end{align}
\end{subequations}

It can be seen that $v_\pi = v_b - v_e$. The small signal output resistance, $r_\pi$, is defined as $\frac{\beta}{g_m}$~\cite[p. 29]{ADAIC}.

Using Kirchoff's current law on the emitter:

\begin{subequations}
\begin{align}
&\frac{v_e - v_b}{r_\pi} - g_m (v_b - v_e) + \frac{v_e}{R_{e2}} + \frac{v_e - v_c}{r_0} = 0\\
\textrm{Assuming $r_0$ is large: } &\frac{v_e - v_b}{r_\pi} - g_m (v_b - v_e) + \frac{v_e}{R_{e2}} = 0 \label{eq:kclEmitter}
\end{align}
\end{subequations}

Using Kirchoff's current law on the collector:

\begin{equation} \label{eq:kclCollector}
\frac{v_c}{R_c} + g_m(v_b - v_e) = 0
\end{equation}

Substituting the definition of $r_\pi$ into equation~\ref{eq:kclEmitter}:

\begin{subequations}
\begin{align}
\frac{(v_e - v_b)g_m}{\beta} - g_m(v_b - v_e) + \frac{v_e}{R_{e2}} &= 0\\
g_m \left( \frac{v_e - v_b}{\beta} + (v_e - v_b) \right) + \frac{v_e}{R_{e2}} &= 0\\
g_m (v_e - v_b) \left( \frac{1}{\beta} + 1 \right) + \frac{v_e}{R_{e2}} &= 0 \label{eq:kclEmitter2}
\end{align}
\end{subequations}

Equating equations \ref{eq:kclEmitter2} and \ref{eq:kclCollector}:

\begin{subequations}
\begin{align}
g_m (v_e - v_b) \left( \frac{1}{\beta} + 1 \right) + \frac{v_e}{R_{e2}} &= \frac{v_c}{R_c} + g_m(v_b - v_e)\\
g_m (v_e - v_b) \left( \frac{1}{\beta} + 2 \right) + \frac{v_e}{R_{e2}} &= \frac{v_c}{R_c} \label{eq:emitterEqCollector}
\end{align}
\end{subequations}

By Ohm's law:

\begin{subequations}
\begin{align}
v_e &= i_e R_{e2}\\
&= R_{e2} (i_b + i_c)\\
&= R_{e2} (i_b + \beta i_b)\\
&= R_{e2} i_b (\beta + 1) \label{eq:v_e}
\end{align}
\end{subequations}

\begin{subequations}
\begin{align}
v_b - v_e &= r_\pi i_b\\
\therefore v_b - R_{e2} i_b (\beta + 1) &= r_\pi i_b
\end{align}
\end{subequations}

\begin{subequations}
\begin{align}
v_c &= -i_c R_c\\
&= -\beta i_b R_c \label{eq:v_c}
\end{align}
\end{subequations}

Substituting equation \ref{eq:v_e} into \ref{eq:emitterEqCollector}

\begin{subequations}
\begin{align}
g_m \left(R_{e2} i_b (\beta + 1) - v_b\right) \left( \frac{1}{\beta} + 2 \right) + i_b (\beta + 1) &= \frac{v_c}{R_c}
\end{align}
\end{subequations}

\section{Voltage Gain Measurement}

\section{Impedance Measurement}

\section{Reflection or w/e}

\begin{appendices}
    \label{appendix}
    \section{Some appendix}
    \label{app:one}
\end{appendices}

\bibliographystyle{IEEEtran}
% IEEEabrv abbreviates journal titles in accordance to IEEE standards 
\bibliography{mybib}

\end{document}